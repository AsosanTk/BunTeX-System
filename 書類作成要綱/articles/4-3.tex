\newpage
\pagestyle{leaflet}
\inserttitle*[running head]{第4回:周辺知識}[ライセンス]
\begin{multicols*}{2}
\section{\BunTeX のライセンス}
\singleimage{74mm}{assets/creative-commons.png}
\BunTeX は\href{https://creativecommons.org/licenses/by-nc-sa/4.0/}{クリエイティブ・コモンズ 表示 - 非営利 - 継承 4.0 国際 ライセンス}の下に提供されています。
以下のライセンスに反しない限り無償で利用できます。
またここでの「\BunTeX 」は、{\BunTeXB}・{\BunTeXC}・{\BunTeXD}と、各種テンプレートからなる
一連のファイル群であると規定します。

\point*[running head]{BY-表示}
文実\uwave{以外}の団体で\BunTeX を使用したい場合は、
\uwave{文書に適切なクレジット(著作者:Asosan)を示す必要があります}。

\point*[running head]{NC-非営利}
いかなる場合であっても、営利目的で\BunTeX を頒布することは認められていません。

\point*[running head]{SA-継承}
今後Lua\LaTeX のバージョンアップ等に合わせて\BunTeX を修正する必要がある場合は、
適切に行って構いません。
ただし加工した場合には、元の\BunTeX と同じライセンスを持ち、原則としてそれに遵守して頒布を行う必要があります。

また、2023/3月までの改変は原則として禁止します。


\section{フォントのライセンス}
1-2でも言及した通り、
配布したフォントの一部にはAsosanのMacから直接引っ張ってきたものがあります。
当然ですが、システムにプリインストールされているフォント
を他者に配布することは認められていません。具体的には以下のフォントです。
\begin{reitemize}
    \item Helvetica Neue(欧)
    \item HiraKakuProN~W1-W9(和/欧)
    \item HiraMaruProN(和/欧)
    \item HiraMinProN(和/欧)
    \item Kyokasho(和)
    \item Palatino(欧)
\end{reitemize}
\boldwave{これらのフォントは、文実で\TeX を用いて書類を作る時以外には絶対に使用しないでください。}

また、これ以外のフォントはフリーフォントになります。以下に一覧を掲載しておきます。
\begin{reitemize}
    \item GenEiNombre
    \item HuiFont
    \item LatinModern
    \item Mikachan
\end{reitemize}
各フォントはそれぞれのガイドラインに従って適切に使用してください。

\end{multicols*}