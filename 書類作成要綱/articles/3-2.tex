\begin{comment}

    
    \newcolumn
    
    \section{規則}
    \hspace{1zw}複数人で執筆を行う際、対策をしないと全ての原稿を合わせた時にフォーマットが所々違う、変な冊子が出来上がってしまいます。
    これを防ぐため、このセクションでは要綱執筆時の統一したルールを書いていきたいと思います。\\
    \indent まだ書いていません。ごめんなさい。そのうちやります。\\
    \begin{comment}
    \subsection{1}{文字のルール}
    
    \subsection{2}{画像の入れ方}
    
    \subsection{3}{リストの使い方}
    
    \subsection{4}{枠囲み}
    
    \subsection{5}{文献参照}
    
    \subsection{6}{索引対象となる用語}

\newcolumn
\section{バージョン管理}
\hspace{1zw}作業要綱は複数人で編集する必要があるので、同時に編集が行えてかつ統一したバージョン管理を行う必要があります。そこで今回使用するのが\textbf{GitHub}です(みなさん一度は聞いたことがあると思います)。\\

\subsection{0}{GitHubって何?}
\hspace{1zw}GitHubは、「Git」というバージョン管理ツールをクラウド上で使用できるようにしたWebサービスです。
「いつ誰がどこを編集したのか」や「最新のバージョンはどれになるのか」などを明確にすることで、要綱の編集をより効率的・スムーズに行うことができます。\\

\subsection{1}{登録にする}
\point{アカウント作成}
\indent まずは各個人でGitHubのアカウントを作成してください。以下のサイト等を参照すれば簡単に作成することができるかと思います。\\
\url{https://qiita.com/ayatokura/items/9eabb7ae20752e6dc79d}\\

\point{Organizationに登録する}
\hspace{1zw}次に作成したアカウントを文実のチームに追加します。招待を送りますので、DiscordにアカウントのIDを貼ってください。\\

\point{リポジトリを見る}
Organizationの中に「WorkOutline」と「PreparationOutline」という2つのリポジトリ(プロジェクト)があります。これが今回使用するものになります。

\point{GitHub Desktopのダウンロード}
\url{https://desktop.github.com}より、「GitHub Desktop」をインストールしておきます。後述するクローン・コミット・プッシュといった作業を自分のPCで行う場合はこれを使用するのが良いでしょう。

\subsection{2}{知識}
GitHubを使う上での前提知識をいくつか解説します。\footnotemark[4]\\

\point{ローカルリポジトリとリモートリポジトリ}
\indent リポジトリは、ファイルやディレクトリの状態を保存するスペースのようなものです。管理したいディレクトリをリポジトリと連携させることで、そのディレクトリ内のファイルの変更履歴を記録し、保存していくことができます。\\
リポジトリは自分のPC内に記録される「ローカルリポジトリ」と、ネットワーク上に存在する「リモートリポジトリ」の2つがあります。ローカルリポジトリで作業を行ったものを、リモートリポジトリへプッシュする流れが基本的な方法となります。\\
\indent なお、リモートリポジトリはGitHub上で作成が可能です。ローカルリポジトリについては、GitHub上で作成したリモートリポジトリをクローンする形で作成するのが一般的です。\\

\point{クローン(clone)}
\indent リモートリポジトリをローカルにダウンロードするコマンドです。その時の最新版のデータと変更履歴などがまとまっていますので、クローンしたタイミングのリモートリポジトリと全く同じ環境をローカルに作成します。これにより、他の人の影響を受けずに自分のPCで作業ができるようになります。\\

\point{ブランチ(branch)}
\indent ブランチとは、作業を分岐させて履歴の流れを保存していく方法のことをいいます。 分岐されたブランチは他のものの影響を受けないので、1つのソフトウエアに対して複数のメンバーが同時に、バグの修正や新たな機能追加を行うことができます。\\
\indent おおもとのデータがあれば、修正されたブランチをマージ(導入や合流)させることで、ファイルを一から作り直すことなくさまざまな修正を行うことが可能です。\\

\point{コミットとプッシュ(commit / push)}
\indent コミットとは、ファイル追加や変更の履歴をリポジトリに記録することです。プッシュはファイル追加や変更の履歴をリモートリポジトリにアップする操作のことをいいます。\\
\indent なお、コミット前に修正したファイルをアッド(add)する必要があります。Gitは、まず仮の保管場所に変更したファイルをまとめ、そこに名前をつけてパッケージにするという手順をとります。この仮の保管場所に保存するコマンドがアッド、名前をつけてパッケージにするコマンドがコミットです。\\

\point{プルリクエスト(Pull request)}
\indent プルリクエストとは、自分が行った変更をオリジナルのものに反映させたいというときに使う通知方法です。オリジナルのオーナーにプルリクエストを通知することができます。\\
\indent なお、プルリクエストの処理は、GitHub上で行うことが可能です。ブランチに対してプッシュを行った場合、GitHubで該当のリポジトリの画面を開くと、「Compare \& pull request」というボタンが出ています。このボタンを押下することで、プルリクエスト作成画面に移行できます。\\

\footnotetext[4]{\url{https://www.modis.co.jp/candidate/insight/column_30}より引用}
\subsection{3}{編集する際にやること}
\point{クローンする(初回のみ)}
\indent GitHub Desktopにサインインし、以下の2つのリポジトリを「クローン」して、自分のPCにおいておきます。
\begin{itemize}
    \item TkBunjitsuOfficial/WorkOutline
    \item TkBunjitsuOfficial/PreparationOutline
\end{itemize}


\point{編集する}
\indent 自分が担当するチャプターのtexファイルを編集していきましょう。\\

\point{更新する(適宜)}
\indent GitHub Desktopの上側のメニューバーにある、「Fetch origin」を適宜押して、他の人が編集した変更を適用してください。\\


\subsection{4}{変更を反映する\footnotemark[5]}
\point{ブランチを作成する(初回のみ)}
\indent GitHub Desktopの上側のメニューバーの「Current Branch」を押して、「New Branch」を選択します。入力欄に「chapterX」(Xは章番号)といれて、各章を編集するための新しいブランチを作成します。\\

\point{コミットしてリモートリポジトリにプッシュする(毎回変更時)}
\begin{enumerate-circle}
    \item \textbf{「Current Branch」の欄が各章のブランチになっているか確認します。}
    \item 左側の欄で全てのファイルにチェックマークが付いていることを確認し、Summaryの部分に変更点の概要を、また必要に応じてdescriptionに説明を記述します。
    \item 一番下の「Commit to chapterX」を押します。これで自分が行った変更を「コミット」することができました。
    \item 一番上のバーの「Push origin」を選択して、リモートリポジトリに「プッシュ」します。
\end{enumerate-circle}

\point{chapter内の編集が完全に終了した時}
\begin{enumerate-circle}
    \item Github上のリポジトリページに飛びます。pushしたブランチのPull Request(PR)を出せるようになっていますので、画面右の[Compare \& pull request]を押してください。
    \item PRを作成する画面に飛ぶので、以下の手順で編集・作成してください。\begin{itemize}
        \item Reviewerを指定する
        \item タイトルを[ブランチ名] 要約とする
        \item 最低限の概要とmergeを受け入れる条件を書く
        \item \hspace{0mm}[Create pull request]を押す
    \end{itemize}
\end{enumerate-circle}
\footnotetext[5]{\url{https://qiita.com/siida36/items/880d92559af9bd245c34}を参考にした。}


\end{comment}