\newpage
\pagestyle{leaflet}
\inserttitle*[running head]{第1回:導入}[1-2 環境構築]
\begin{multicols*}{2}
この章では、\TeX 本体、\BunTeX スタイルファイル、さらにはフォントファイル等のインストール方法について解説していきます。
以下の手順でファイル一式をインストールすることで、自分のパソコンで資料やレポートを作れる環境が整います。
\begin{framebox-simple}{おしながき}
\begin{enumsquarebrackets}
    \item \TeX のインストール\footnotemark[1]
    \item Visual Studio Codeの導入
    \item \BunTeX スタイルファイルのインストール
    \item その他のファイル
    \item GitHub
\end{enumsquarebrackets}
\end{framebox-simple}
\footnotetext[1]{\url{https://qiita.com/passive-radio/items/623c9a35e86b6666b89e}を参照しました。}
\section{\TeX のインストール}
まずはあらゆる\TeX ファイルをパソコンでコンパイルするために必要となる環境構築を行います。
なお、WindowsとMacで多少手順に違いがありますので、自分のパソコンに合わせて選択してください。

\point*[running head]{Windows: TeX Liveをインストールする}
ダウンロードページ(\url{https://www.tug.org/texlive/})にアクセスして「TeX Live 2022」をダウンロードし、
表示されるダイアログに従ってインストールしてください。ファイルサイズが4.5GBと非常に大きいので、事前に本体のストレージに余裕があるかを確認しておくようにしてください。

\point*[running head]{Mac: MacTeXをインストールする}
ダウンロードページ(\url{https://www.tug.org/mactex/})にアクセスして「MacTeX-2022」をダウンロードし、
表示されるダイアログに従ってインストールしてください。

\point*[running head]{動作チェック}
インストールには3時間ほど要します。気長に待ってください。\\
\indent インストールが終わったら、パソコンのメニューから「コマンドプロンプト」(Macの場合はターミナル)を起動し、
「lualatex --version」と打ち込んで実行します。画面にエラーが出ず、バージョン名等がでてきたらインストール成功です。

\section{Visual Studio Codeの導入}
前のセクションで\TeX をコンパイルすることができるようになりましたが、このままでは毎回の書類作成時に
「メモ帳を開いて、編集して、コマンドプロンプトで実行」という手順を踏まなければいけないので非常に不便です。
そこでVisual Studio Code(以下ではVSCodeと省略)というエディタを導入することで、気軽に\TeX を書く作業が
できるようになります。
\point*[running head]{VSCodeのダウンロード}
公式サイト(\url{https://code.visualstudio.com/download})から
自分のパソコンにあったバージョンを選択してダウンロード\& インストールします。

\point*[running head]{\LaTeX 拡張機能を入れる}
次に拡張機能を入れていきます。
VSCodeを起動したら、左側のメニューバーの上から5番目にある「Extensions」を選択します。
検索窓から「Japanese Language Pack」と「LaTeX Workshop」を検索してインストールしてください。\\
終わったら一度アプリの再起動を行っておきましょう。
\point*[running head]{\LaTeX 拡張を設定する}
先ほどの「拡張機能」の画面に移動し、以下の写真で示す手順で「settings.json」を開きます。
\singleimagecap{70mm}{assets/1-2vs1.png}{赤い部分を順に選択してください}
\singleimagecap{70mm}{assets/1-2vs2.png}{赤い丸を押すとsetting.jsonが開きます}
そこにGitHubにある「vssetup.txt」のコードをコピー&ペースト
してください。


\section{\BunTeX スタイルファイルのインストール}
\point*[running head]{GitHubからインストール}
文実GitHub(\url{https://github.com/Asosan/BunTeX-System})
の右上のCodeボタン(緑色)から、zipファイルをダウンロードします。

\point*[running head]{スタイルファイルの配置}
zipを解凍してファイルを開き、「Resources/BunTeX」と「Resources/Style」以下にある
全てのスタイルファイルを、自分のパソコンの「/usr/share/texlive/texmf-dist」(Macの場合は/usr/local/texlive/texmf-local/tex/latex)に配置します。

\point*[running head]{反映させる}
コマンドプロンプト / ターミナルから、「sudo mktexlsr」を実行して\TeX 本体に行われた変更(今回はスタイルファイルの追加)の
反映を行います。終わったらパソコンを再起動しておきましょう。


\section{その他のファイル}
\point*[running head]{フォントファイル}
資料テンプレートは、さきほどGitHubからダウンロードしたファイルの「Resources/Fonts」
の中にあります。

Windowsの場合は、パソコンのメニューから設定→個人用設定→フォントの順に選択して、すべてのフォントファイルをドラック{\&}ドロップします。

Macの場合は、Font Bookの中に、まだパソコンの中にないフォントをドラックします(例えばヒラギノ系統はデフォルトでインストールされているので必要ありません)。

これらのフォントの一部にはAsosanのMacから引っ張ってきたものがあります。当然ですが、システムにプリインストールされているフォント
を他者に配布することは認められていません。\boldwave{これらのフォントは、文実で\TeX を用いて書類を作る時以外には絶対に使用しないでください。}
\point*[running head]{資料テンプレート}
「Resources/Templates」の中に入っています。
各資料の説明は後の章で行います。


\section{GitHub}
GitHubとは、「Git」というバージョン管理ツールをクラウド上で使用できるようにしたWebサービスです。
「いつ誰がどこを編集したのか」や「最新のバージョンはどれになるのか」などを明確にすることができ、大変便利です。

GitHubを使用したい場合は、公式サイト(\url{https://github.com})よりアカウントを作成し、
作成後に文実用GitHub組織「TkBunjitsuOfficial」に参加してください。
既に組織に所属している人に招待してもらう必要があります。
\sectiont{メモ}
  
\vspace{-20mm}
\writeborder[height fill]{}
\end{multicols*}
