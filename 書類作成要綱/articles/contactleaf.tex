\pagestyle{empty}
\contactleaf*[running head]{書類作成要綱 概要}

\begin{multicols*}{2}
\setlength{\columnseprule}{0pt}
\contactleafsection*[running head]{概要}
この解説冊子は、今年度より導入された\TeX を使用した資料・要綱の作り方について解説していきます。

2021年度2月より試行錯誤を繰り返してきた\BunTeX ですが、3ヶ月の時を経てついに完成版を出すことができました。
大変長らくお待たせしました。

まずはみなさんが\TeX に慣れ、以前のWordの半分の時間で
効率よく資料が作成できるようになることを目標にしてほしいと思います。
日本語が崩壊していて読みにくい箇所も多々あるかとは思いますが、どうか最後までお読みいただけると幸いです。


\contactleafsection*[running head]{目次}
\begin{enumsquare}
    \setlength{\leftskip}{-0.7\zw}
    \item 導入
    \begin{enumerate}
        \renewcommand{\labelenumii}{\hspace{4\zw}□~\ajKakko{\arabic{enumii}} }
        \item \TeX の基本
        \item 環境構築
        \item \BunTeX とは
    \end{enumerate}
    \item 資料用\BunTeX
    \begin{enumerate}
        \renewcommand{\labelenumii}{\hspace{4\zw}□~\ajKakko{\arabic{enumii}} }
        \item 共通事項
        \item 基本書類
        \item 解答用紙
        \item リーフレット
    \end{enumerate}
    \item 冊子用\BunTeX
    \begin{enumerate}
        \renewcommand{\labelenumii}{\hspace{4\zw}□~\ajKakko{\arabic{enumii}} }
        \item 要綱・本
        \item 複数人で執筆する際は
    \end{enumerate}
    \item 周辺知識
    \begin{enumerate}
        \renewcommand{\labelenumii}{\hspace{4\zw}□~\ajKakko{\arabic{enumii}} }
        \item 作成上の注意
        \item 各種テンプレート
        \item ライセンス
    \end{enumerate}
\end{enumsquare}
\noindent ※ この冊子は\lastpage ページで構成されています。乱丁・落丁がないか今一度ご確認ください。

\contactleafsection*[running head]{本書の使い方}
本書は全4回分で構成されています。

\contactleafsection*[running head]{テスト}
復習テスト・確認テスト・総復習テスト等のテストは実施しませんが、
4回を通して少しでも早く書類を作成できるようになることを目標とします。

\contactleafsection*[running head]{ノート}
この冊子は書き込み形式になっています。必要に応じてメモを取るようにしてください。
別にノートを用意する必要はありません。

\contactleafsection*[running head]{宿題}
\noindent ⑴ \boldtext{パソコンの環境を整える}\\\indent 第1回の宿題は、自分のパソコンで\TeX がかける環境を整えることです。\\

\noindent ⑵ \boldtext{冊子内容の復習}\\\indent 第2〜3回の宿題は、\BunTeX の機能を知り、自力で書けるようになることです。\\

\noindent ⑶ \boldtext{実践}\\\indent 第4回の終了後は、各自資料を作っていきましょう。\\


\contactleafsection*[running head]{作成者}
\BunTeX 及びこの冊子の作成者は、72期 麻生(Asosan)です。
ご意見・修正依頼等は、Gafeメール(72.asoo.kosei@tsukukoma-gafe.org)までお願いします。
またわからないことがあれば、いつでも文実Discordの【\# 02a-tex-system】チャンネルにて質問してください。

\end{multicols*}