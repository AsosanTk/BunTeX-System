\newpage
\pagestyle{leaflet}
\inserttitle*[running head]{第2回:資料用{\BunTeX}}[共通事項]
\begin{multicols*}{2}
ここからは資料用\BunTeX の使い方について解説していきます。\mline{なお全て書くのは非常に手間なので}、一部以下のように略して掲載します。

\begin{framebox-simple}{省略記号一覧}
\noindent\verb|[A]|:\verb|\<コマンド名>{<文字列>}|\\
\noindent\verb|[B]|:\verb|\begin{<命令の名前>}[<オプション>]{<情報1>}...|\\\hspace{4.5\zw}\verb|<コンテンツ>|\\\hspace{2.4\zw}\verb|\end{<命令の名前>}|\\
\noindent\verb|[C]|:\verb|\<コマンド名>*[running head]{<文字列>}|\\
\noindent\verb|[D]|:\verb|\<コマンド名>|\\
\noindent\verb|[E]|:\verb|\<コマンド名>{<情報1>}{<情報2>}...|
\end{framebox-simple}
\noindent ※ <>の部分を変えていく感じです。\\
\noindent ※ <文字列>には任意の文字列を入れてください。\\
\noindent ※ <コンテンツ>には複数行にわたり、他のコマンドを含むものなどが入ります。\\
\noindent ※ 専門用語で\verb|[A]|のことをマクロ、\verb|[B]|のことを環境と呼びます。\\
\noindent ※ \verb|*[running head]|はつけなくても良い。

\section{種類}
資料用\BunTeX で作れる書類は以下の通りです。
\begin{framebox-simple}{作成可能な書類一覧}
    \begin{reitemize}
        \item 定例会資料
        \item 生徒部提出資料・掲示資料
        \item 解答用紙
        \item 冊子~/~リーフレット(デコ責会議等)
    \end{reitemize}
\end{framebox-simple}
\indent ※ デコ責会議の冊子は、各部門が配布する資料類を一冊にまとめたものです。

\section{前提}
コンパイルは「Lua\LaTeX 」で行います。プリアンブル(\TeX の基本 - セクション3参照)
には以下を記述し、buntexc, buntexdの2種類のスタイルファイルを読み込みます。
\begin{framebox-simple}{プリアンブルの記述方法}
\begin{verbatim}
\documentclass[<options>]{jlreq}

\usepackage{buntexc}
\usepackage{buntexd}

\begin{document}
    <本文>
\end{document}

\end{verbatim}
※ \verb|<options>|の部分はテンプレートによって違うものが入る
\end{framebox-simple}
\point{改行}
\TeX ファイル内で1行空白行を挟むと改行されます。ただし2行以上空白にしたとしても、出力されるのは改行のみで、空白行は出力されません。
空白行を挿入する以外に文章中で強制的に改行したい場合は、その箇所に「\verb|\\|」を記述してください。

\point{余白}
余白は縦・横方向それぞれ次のような命令があります。
\begin{framebox-simple}{横余白}
\begin{reitemize}
    \item \verb|~|:半角空白
    \item  :全角空白(日本語キーボードでスペースを押す)
    \item \verb|\hspace{<長さ>}|
    \item \verb|\hskip<長さ>|
    \item \verb|\hfill|:空白を横に限界まで伸ばす
\end{reitemize}
※ 半角空欄は半角スペースを入れることによっても出力できますが、複数の半角スペースを打っても、出力されるのは1つだけです。
\end{framebox-simple}
\begin{framebox-simple}{縦余白}
    \begin{reitemize}
        \item \verb|\vspace{<長さ>}|
        \item \verb|\vskip<長さ>|
        \item \verb|\vfill|:空白を縦に限界まで伸ばす
    \end{reitemize}
\end{framebox-simple}
また長さの単位は以下の通りです。長さには自然数だけでなく、負の数や小数を指定することもできます。
\begin{framebox-simple}{単位}
    \begin{reitemize}
        \item mm, cm:お馴染みのSI単位系
        \item \verb|\zw|:和文文字の横幅と同じ
        \item \verb|\zh|:和文文字の縦幅と同じ
    \end{reitemize}
\end{framebox-simple}

\section{見出し}
見出しは3段階に分かれています。階層に応じて以下のように指定してください。
\begin{framebox-simple}{見出し1}
    \begin{reitemize}
        \item \verb|[A]| section:番号がつきます
        \item \verb|[C]| sectionb
        \item \verb|[C]| sectionc
        \item \verb|[C]| sectiond
        \item \verb|[C]| sectiont
    \end{reitemize}
    \checking{b〜d, tは番号がつきません。若干デザインが変わるだけなので気分で選んでください。}
\end{framebox-simple}
\begin{framebox-simple}{見出し2}
    \verb|[C]| point:▶︎がでて太字下線になります。
\end{framebox-simple}
\begin{framebox-simple}{キーポイント}
    \verb|[A]| checking:▷がついて若干インデントが変わります。
\end{framebox-simple}

\section{テキスト装飾}
テキストの一部を強調したり、テイストを変えたい場合に使用するコマンド一覧です。フォントのデフォルトはヒラギノ明朝~/~ヒラギノゴシック~/~Palatinoです。
\begin{framebox-simple}{スタイル}
    \begin{reitemize}
        \item \verb|[A]| gothic:囲んだ部分をゴシックに
        \item \verb|[A]| boldtext:囲んだ部分を太字に
        \item \verb|[A]| uline:アンダーライン
        \item \verb|[A]| uwave:波線
        \item \verb|[A]| mline:取り消し線
        \item \verb|[A]| boldwave:太字かつ波線
        \item \verb|[A]| highlighter:マーカーを引きます
    \end{reitemize}
\end{framebox-simple}
\begin{framebox-simple}{フォントサイズ}
\verb|\selectsize{<フォントサイズ>}{<行送り>}|\\

\noindent ※ 単位はptです
\end{framebox-simple}
\begin{framebox-simple}{和文フォント}
    \begin{reitemize}
        \item \verb|[A]| HiraMaruJ:ヒラギノ丸ゴシック
        \item \verb|[A]| HiraKakusix:ヒラギノゴシックW6(極太字)
        \item \verb|[A]| HuiFont:ふい字フォント
        \item \verb|[A]| Kyokasyo:游教科書体
    \end{reitemize}
\end{framebox-simple}
\begin{framebox-simple}{欧文フォント}
    \begin{reitemize}
        \item \verb|[A]| LatinModern:\TeX 標準
        \item \verb|[A]| Helvetica
        \item \verb|[A]| Hiramin:ヒラギノ明朝
        \item \verb|[A]| HiraMaru:ヒラギノ丸ゴシック
        \item \verb|[A]| HiraKakusixE:ヒラギノゴシックW6(極太字)
        \item \verb|[A]| GenEiNombre:ノンブル(ページ番号)用フォント
    \end{reitemize}
\end{framebox-simple}

\section{リスト}
\verb|[B]|の\dots の部分に\verb|\item|で要素をどんどん並べていくとリストになります。
\begin{framebox-simple}{リスト}
    \begin{reitemize}
        \item \verb|[B]| reitemize:itemizeのインデントを修正したもの
        \item \verb|[B]| enumbrackets:⑴⑵...
        \item \verb|[B]| enumsquarebrackets:□⑴のように番号にチェックボックスがつく
        \item \verb|[B]| enumcircle:①②...
        \item \verb|[B]| enumsquare:□囲みの数字
    \end{reitemize}
\end{framebox-simple}
\begin{framebox-ref}{リストの使用例}
上の枠囲みの中のリストのコード例を掲載します。
\begin{verbatim}
\begin{reitemize}
    \item [B] reitemize:itemizeのインデントを修正したもの
    \item [B] enumbrackets:⑴⑵...
    \item [B]| enumsquarebrackets:□⑴のように番号にチェックボックスがつく
    \item [B]| enumcircle:①②...
    \item [B] enumsquare:□囲みの数字
\end{reitemize}
\end{verbatim}
\end{framebox-ref}

\section{枠囲み}
これを使用すると、コンテンツ全体を枠で囲うことができるようになります。本文から分けたり、強調したり
といった使い方ができます。
\begin{framebox-simple}{枠囲みの種類}
    \begin{table}[H]
        \begin{tabularx}{64mm}{|X|c|c|}
        \hline
        命令の名前             & 情報1  & オプション  \\ \hline
        framebox-simple   & タイトル &        \\ \hline
        framebox-simpled  & タイトル & サブタイトル \\ \hline
        framebox-key      &      &        \\ \hline
        framebox-ref      & タイトル &        \\ \hline
        framebox-brackets & タイトル & サブタイトル \\ \hline
        framebox-practice & タイトル &        \\ \hline
        \end{tabularx}
    \end{table}
    \begin{reitemize}
        \item \verb|[B]| framebox-simple:単純な枠囲み
        \item \verb|[B]| framebox-simpled:上にサブタイトルを追加したもの
        \item \verb|[B]| framebox-key:鉄○会の【KEY】
        \item \verb|[B]| framebox-ref:参考
        \item \verb|[B]| framebox-brackets:鉤括弧風の枠
        \item \verb|[B]| framebox-practice:練習問題
    \end{reitemize}
\end{framebox-simple}

\section{図・画像}
図や画像を入れたい場合は、以下を指定します。
\point{通常の画像}
2枚横並びまであります。2段組の場合は、1枚画像(70mm)を指定することを推奨します。
\begin{framebox-simple}{1枚画像}
    \begin{itemize}
        \item \verb|[E]| \verb|singleimage{<横幅>}{<画像のパス>}|:通常
        \item \verb|[E]| \verb|singleimagecap{<横幅>}{<画像のパス>}{<説明>}|:キャプション付
    \end{itemize}
\end{framebox-simple}
\begin{framebox-simple}{2枚画像}
    \begin{itemize}
        \item \verb|[E]| \verb|doubleimage{<横幅1>}{<画像1のパス>}{<画像2のパス>}|:通常
        \item \verb|[E]| \verb|doubleimage{<横幅1>}{<画像1のパス>}{<説明1>}{<画像2のパス>}{<説明2>}|:キャプション付
    \end{itemize}
\end{framebox-simple}
\point{QRコード}
QRコードは3枚横並びまでできます。Footnote等でリンクを貼ることを強く推奨します。
\begin{framebox-simple}{QRコード}
    \begin{itemize}
        \item \verb|[E]| \verb|singleqr{<画像のパス>}{<説明>}|
        \item \verb|[E]| \verb|doubleqr{<画像1のパス>}{<説明1>}{<画像2のパス>}{<説明2>}|
        \item \verb|[E]| \verb|tripleqr{<画像1のパス>}{<説明1>}{<画像2のパス>}{<説明2>}{<画像3のパス>}{<説明3>}|
    \end{itemize}
\end{framebox-simple}

\section{表}
\LaTeX の表は、自分で一から書くと非常に手間がかかります。ここでは外部サイトを用いて簡単に表を作成する方法を紹介します。
\point*[running head]{表を作成する}
Tables Generator(\url{https://www.tablesgenerator.com/latex_tables#})を使用して表を作成していきます。メニューから\mline{表のサイズ・枠線}・テキスト装飾
等を各セル個別に設定することができます(もちろん選択の仕方次第で表全体にスタイルを適用することもできます)。
\singleimagecap{70mm}{assets/2-1table1}{Tables Generatorの編集画面}
\point*[running head]{修正}
Fig 2.2のような表のコードが生成されたらこれをコピーして該当箇所に貼り付けます。
ただしこのままでは使うことができず\footnotemark[2]、以下のように軽微な修正が必要となります。
\singleimagecap{70mm}{assets/2-1table2}{Tables Generator}
\begin{enumcircle}
    \item 赤枠の部分に「H」と打ち込みます。
    \item 緑枠の部分の「tabular」を「tabularx」に変更します。
    \item 紫の部分に\verb|{<表の横幅>}|を追加します。二段組の場合は基本的に70mmで、枠囲み内では64〜66mm、一段組は状況に応じて最大163mmを指定すると良いでしょう。
    \item 青枠の部分で表の行寄せを変更します。
\end{enumcircle}
\footnotetext[2]{一応そのままでも表は出力されますが、ダサいです。tabularよりは横幅を指定できるtabularxの方が任意性が高いので、基本的には手順に従うようにしてください。}
\begin{framebox-ref}{行寄せのコマンド}
青枠に指定するコマンドについて解説します。
\begin{reitemize}
    \item l:左寄せ・幅は1番長い文字列に合う
    \item c:中央寄せ・同上
    \item r:右寄せ・同上
    \item L:左寄せ・幅はl,c,rで指定したものを除いて、指定された表の幅内で均等に長さをとって配置します。
    \item C:中央寄せ・同上
    \item R:右寄せ・同上
    \item |:区切り線を引く・連続で並べると二重線, 三重線...になる
\end{reitemize}
\end{framebox-ref}
\begin{framebox-ref}{表の一例}
上の「枠囲みの種類」の中で使用した表のコードを掲載しておきます。\\

{\selectsize{6pt}{7pt}
\begin{verbatim}
    \begin{table}[H]
        \begin{tabularx}{64mm}{|X|c|c|}
        \hline
        命令の名前             & 情報1  & オプション  \\ \hline
        framebox-simple   & タイトル &        \\ \hline
        framebox-simpled  & タイトル & サブタイトル \\ \hline
        framebox-key      &      &        \\ \hline
        framebox-ref      & タイトル &        \\ \hline
        framebox-brackets & タイトル & サブタイトル \\ \hline
        framebox-practice & タイトル &        \\ \hline
        \end{tabularx}
    \end{table}
\end{verbatim}
}
\end{framebox-ref}


\section{その他}
\begin{framebox-simple}{注釈}
    \begin{reitemize}
        \item \verb|\footnotemark[N]|:注釈をつけたいテキストの直後に記述する~/~Nには自然数が入る
        \item \verb|\footnotetext[N]{<説明>}|:注釈を書いて出力~/~Nにはfootnotemarkに対応する自然数が入る
    \end{reitemize}
\end{framebox-simple}
\begin{framebox-simple}{URL}
    \begin{reitemize}
        \item \verb|\url{<URL>}|:直接URLを出力する~/~注釈で使うと良い
        \item \verb|\href{<URL>}{<テキスト>}|:テキストにリンクを貼る(直接URLを見ることはできない~/~非推奨)
    \end{reitemize}
\end{framebox-simple}
\begin{framebox-simple}{書類末に使用}
    \begin{reitemize}
        \item \verb|[D]| \verb|responsibility{<責任者名>}{<所属>}|:文責(所属には慣例的に部門名が用いられる)
        \item \verb|[D]| \verb|allresponsibility{<責任者名>}{<所属>}|:統括文責(同上)
        \item \verb|[D]| \verb|distribution{<日付>}|:配布日付
        \item \verb|[D]| lastpage:書類のページ総数
    \end{reitemize}
\end{framebox-simple}
\begin{framebox-simple}{ロゴ等}
    \begin{reitemize}
        \item \verb|[D]| snowman:雪だるまの文字
        \item \verb|[D]| BunTeX:メインロゴ
        \item \verb|[D]| BunTeXC, BunTeXB, BunTeXD
        \item \verb|[D]| BunTeXJ:日本語版のロゴ
    \end{reitemize}
\end{framebox-simple}
\begin{framebox-ref}{URLの使用例}
\begin{verbatim}
筑駒文化祭\footnotemark[1]は、毎年10月か11月に開催されています。
当日はたくさんのJK・JDが来場します。
...
\footnotetext[1]{\url{https://tsukukoma.bunkasai.info}を参照のこと}
\end{verbatim}
\end{framebox-ref}

\section{付番方法}
今年度より、生徒側の書類管理効率化のため、\uwave{対外に出す資料にのみ}付番をすることになりました。
以下にそのルールを挙げます。
\begin{framebox-simple}{デコ責会議資料に対する付番}
\begin{center}
{\selectsize{15pt}{15pt}Dnn-ttxxp}
\vspace{0.5\zw}
\end{center}
\begin{reitemize}
    \item 先頭の「D」は、デコ責で配布された資料であることを示します。
    \item 「nn」には、第N回デコ責会議の「\(N \in \mathbb{N}\)」が2桁で入ります。
    \item 「tt」には、各部門の識別番号が入ります。
    \item 「xx」には、同じ回で同じ部門から配布された資料に対して順番につけます。基本的に若い番号から、説明用資料→提出物説明資料の順(同じ場合は重要度順)に番号を振ってください。
    \item 解答用紙の場合は「p」の部分に「A」と書きます。他の場合には特に指定する必要はありません。
\end{reitemize}
\end{framebox-simple}
\checking{例1:第2回デコ責会議 総務部門 / デコ責広報 / 第5回総務配布資料の中で1枚目の場合→D05-0101}\\
\checking{例2:第7回デコ責会議 器材部門 / 特殊器材について / 第7回器材配布資料の中で4枚目の場合→D07-0604}\\
\checking{例3:第12回デコ責会議 SCC部門 /「当日動線について」に対応する\uwave{解答用紙} / 第12回SCC配布資料の中で2枚目の場合→D12-1102\textbf{A}}
\begin{framebox-ref}{各部門の番号}
    \begin{table}[H]
        \begin{tabularx}{64mm}{|c|C||c|C|}\hline
        ~番号~ & 部門名 & ~番号~ & 部門名 \\\hline
        01 & 総務  & 07 & 電波  \\\hline
        02 & 庶務  & 08 & 広報  \\\hline
        03 & 財務  & 09 & 企画  \\\hline
        04 & 印刷  & 10 & 審査  \\\hline
        05 & 器材  & 11 & SCC \\\hline
        06 & 電力  & 12 & 装飾 \\\hline
        \end{tabularx}
    \end{table}
    \noindent ※ Discordの番号とは異なります。
\end{framebox-ref}

\end{multicols*}
