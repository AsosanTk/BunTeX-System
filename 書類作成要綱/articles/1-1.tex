\newpage
\pagestyle{leaflet}
\inserttitle*[running head]{第1回:導入}[1-1 \TeX の基本]
\begin{multicols*}{2}
【概要】の部分で従来の文実Wordフォーマットを\TeX に変えると宣言しましたが、
これを読んでいる皆さんの中には
「\TeX ...?\TeX って何?(kRd)」状態になっている方もいるかと思います。

この章では、「\TeX の概要」について解説していきます。

\section{概要}
TeX(テフ, テック, \mline{蝶})は、スタンフォード大の数学者であるDonald Ervin Knuth氏
が製作した\uwave{組版システム}です。「組版」とは印刷用語で、活字を組んで版(印刷用データ)を作ることを意味します。決してプログラミング言語ではありません。

\TeX には次のような特徴があります。
\begin{framebox-simple}{\TeX の特徴}
    \begin{reitemize}
        \item オープンソースソフトなので、無料で入手でき、自由に中身を調べたり改良したりできます。また商用利用等も自由にできます。
        \item \TeX は、WindowsでもMacやLinuxなどのUnix系OSでも全く同じ動作をするので、入力が同じなら原理的には全く同じ出力が得られます。
        \item \TeX への入力はテキスト形式なので、普通のテキストエディタで読み書きでき、再利用・データベース化が容易です。
        \item 自動ハイフネーション・ペアカーニング(欧文文字幅を自動的に調整)・リガチャ(fiなどの合字処理)・孤立行処理等の高度な組版技術が組み込まれています。
        \item 特に数式の表記に優れています。
        \item \TeX ファイル単体では数kB程度しかなく、非常に軽いです。Discordの8MB送信制限にも問題なく対応できます。
    \end{reitemize}
\end{framebox-simple}

\section{処理・出力方式}
\TeX のテキストファイルを作成したら(これは次項で解説します)、このファイルを「出力」してPDFに変換します。
出力する際に働くのが「エンジン」です。\LaTeX には次のような種類があります。
\begin{reitemize}
    \item p\LaTeX :長年日本語文書に使用されていた標準
    \item up\LaTeX :上がUnicode文字に対応
    \item Lua\LaTeX :新しいタイプ①
    \item pdf\LaTeX :新しいタイプ②
\end{reitemize}
\vspace{1\zw}
各エンジンは内部で次のような処理を行っています。
\begin{reitemize}
    \item p, up:\TeX ファイル → dviファイル → PDF等
    \item Lua, pdf:\TeX ファイル → PDF(ダイレクト)
\end{reitemize}
\BunTeX で使用するエンジンは、\boldtext{Lua{\LaTeX}} です。

\section{\TeX ファイルの基本}
\point*[running head]{構造}
\begin{verbatim}
①\documentclass[<options>]{jlreq}

②<プリアンブル>

③\begin{document}
    
<本文>

③\end{document}
\end{verbatim}
\vspace{1\zw}
\TeX ファイルの構造は上のようになっています。
\begin{enumcircle}
    \item documentclass:その書類の基本設定を指定します(例えば本文の文字サイズや書類の大きさなど)。\verb|<options>|の部分にはそれぞれの作成物によって違ったものが入ります。\BunTeX ではjlreqというクラスファイルを使用しますが、他に日本語標準のものとしてjsarticle等があります。
    \item プリアンブル:スタイルファイル(後述)の読み込み、自作命令(マクロという, 後述)の定義等を行う場所です。
    \item 本文:begin document, end documentの2つで囲まれた領域に、書類の本文を記述していきます。
\end{enumcircle}

\point*[running head]{テキストを打つ}
\TeX ファイルにただ文字を打っていくだけで通常の文章を打つことができます。

Googleドキュメントの初期画面と同じ状態です。

\point*[running head]{命令}
書類には文章以外にも、タイトルやリスト、テキストの装飾(下線等)を使ったり、画像を貼ったりする必要があります。
これらを実現するには、\TeX 固有の「命令」を使います。
命令には以下の2種類のタイプがあります。
\begin{framebox-simple}{命令の種類}
\begin{reitemize}
    \item 環境:リストなど特定の範囲のレイアウトを変更したいとき、これを実現するために必要となる環境と呼びます。使用するときは該当箇所を\verb|\begin{<環境名>}|と\verb|\end{<環境名>}|で囲います。
    \item コマンド:「環境」以外の命令のことで、\verb|\<コマンド名>{<情報1>}{<情報2>}...|のようにして記述します。
\end{reitemize}
\end{framebox-simple}
Googleドキュメントのメニューバーでデザインを施すのと同じ要領です。\TeX はこれを全てテキスト状態で書くことができるのです。

またこれらの命令をまとめたものを\boldwave{スタイルファイル}といい、\BunTeX はこのスタイルファイルの一種になります(後述)。

\section{\TeX でPDF書類を作成する}
\TeX で書類を作成する流れは以下の通りです。
\begin{enumcircle}
    \item VSCodeを開く(インストールについては1-2参照)
    \item VSCodeでファイルを作りたいフォルダを選択して開く
    \item 拡張子が.texとなるファイルを作成する(これを\TeX ファイルという)(Fig1.1参照)
    \item ③で作成した\TeX ファイルに文章を書いて書類を作る
    \item Lua\LaTeX を用いて「出力」を行う(下の枠囲みを参照) → \TeX ファイルと同じフォルダにPDFが生成される
    \item ④⑤を繰り返し行って仕上げていく(VSCodeで\TeX ファイルとPDFを横並びにしておくとわかりやすいだろう)
    \item 文章中の間違いやレイアウトのミスがないか校閲を行う
    \item 完成
\end{enumcircle}
\singleimagecap{30mm}{assets/1-1vs1.png}{左の丸がファイル作成, 右の丸がフォルダ作成}
\begin{framebox-simple}{VSCodeの\LaTeX 操作パネルの使用方法}
\begin{enumbrackets}
    \item 出力する際に発生する、.auxや.log等の不要なファイルを一括削除
    \item 途中で出力を強制停止させる
    \item pdf\LaTeX の出力
    \item up\LaTeX の出力(\BunTeX では使用しない)
    \item Lua\LaTeX の出力
    \item Lua\LaTeX の出力を2回行う
\end{enumbrackets}
\singleimage{66mm}{assets/1-1vs2.png}
\end{framebox-simple}


\section{エラーが起きた時は}% 2.6
\TeX ファイルの出力の際にエラーが発生する場合があります。VSCode下側の色付きバーの部分に、❌が出たら出力失敗です(通常はチェックマークがでます)。
VSCode下側のOUTPUTを選択すると、logファイルと呼ばれる、Lua\LaTeX エンジンが動いた際の動作ログを確認することができます。多くの場合、このログの最後の方に「!...」のようにしてエラーが出てきます。

主なエラーの原因は以下の通りです。エラーが出た際は落ち着いてひとつひとつ見直していきましょう。
\point{定義されていない命令を使っている}
\BunTeX または\LaTeX で定義されていない命令を使用している可能性があります。命令のスペルミスによるものが多いです。
\checking{解決方法:VSCodeで赤く波線が引かれた部分の周辺が該当箇所です。この資料を参照して命令名が合っているか確認してください。}
\point{括弧の数があっていない}
命令の括弧の数があっていない場合もエラーが起きやすいです。
\checking{解決方法:括弧の対応関係を確認するほかありません。}

また\TeX はどの行でエラーになっているかがわからない場合があります。その場合エラー箇所を探すのに苦労することになりますので、こまめに出力を行うようにしてください。

\section{もっと\TeX を知りたい!}
今まで紹介してきた知識は\TeX のほんの断片的な情報に過ぎません。もっと詳しく\TeX について知りたい、あるいは数学式化学式を用いたレポートを
作成したいなどの場合には、「改訂第8版 - LaTeX2ε美文書作成入門(奥村 春彦先生著)」を適宜参照するとよいでしょう。

ほとんどの基本的な項目をカバーしている本です。文実上本部にもおいてありますが、1冊買って手元に置いておくと便利です。

\end{multicols*}