\newpage
\pagestyle{booklet}
\inserttitle*[running head]{第1回:導入}[1-1 \TeX の基本]
\begin{multicols*}{2}
先ほど【概要】の部分で従来の文実Wordフォーマットを\TeX に変えると宣言しましたが、
これを読んでいる皆さんの中には
「\TeX ...?\TeX って何?(kRd)」状態になっている方もいるかと思います。

まずこの章では、「\TeX とは一体何なのか」について解説していきます。

\section{概要}
TeX(テフ, テック, \mline{蝶})は、スタンフォード大の数学者であるDonald Ervin Knuth氏
が製作した\uwave{組版システム}です。「組版」とは印刷用語で、活字を組んで版(印刷用データ)を作ることを意味します。

\TeX には次のような特徴があります。
\begin{framebox-simple}{\TeX の特徴}
    \begin{reitemize}
        \item オープンソースソフトなので、無料で入手でき、自由に中身を調べたり改良したりできます。また商用利用等も自由にできます。
        \item \TeX は、WindowsでもMacやLinuxなどのUnix系OSでも全く同じ動作をするので、入力が同じなら原理的には全く同じ出力が得られます。
        \item \TeX への入力はテキスト形式なので、普通のテキストエディタで読み書きでき、再利用・データベース化が容易です。
        \item 自動ハイフネーション・ペアカーニング(欧文文字幅を自動的に調整)・リガチャ(fiなどの合字処理)・孤立行処理等の高度な組版技術が組み込まれています。
        \item 特に数式の表記に優れています。
    \end{reitemize}
\end{framebox-simple}

\section{処理・出力方式}
\TeX のテキストファイルを作成したら(これは次項で解説します)、このファイルを「出力」してPDFに変換します。
出力する際に働くのが「エンジン」です。\LaTeX には次のような種類があります。
\begin{reitemize}
    \item p\LaTeX :長年日本語文書に使用されていた標準
    \item up\LaTeX :上がUnicode文字に対応
    \item Lua\LaTeX :新しいタイプ①
    \item pdf\LaTeX :新しいタイプ②
\end{reitemize}
\vspace{1\zw}
各エンジンは内部で次のような処理を行っています。
\begin{reitemize}
    \item p, up:\TeX ファイル → dviファイル → PDF等
    \item Lua, pdf:\TeX ファイル → PDF(ダイレクト)
\end{reitemize}
\BunTeX で使用するエンジンは、\boldtext{Lua{\LaTeX}} です。

\section{\TeX ファイルの基本}
\point*[running head]{構造}
\begin{verbatim}
①\documentclass[<options>]{jlreq}

②<プリアンブル>

③\begin{document}
    
<本文>

③\end{document}
\end{verbatim}
\vspace{1\zw}
\TeX ファイルの構造は上のようになっています。
\begin{enumcircle}
    \item documentclass:その書類の基本設定を指定します(例えば本文の文字サイズや書類の大きさなど)。\verb|<options>|の部分にはそれぞれの作成物によって違ったものが入ります。\BunTeX ではjlreqというクラスファイルを使用しますが、他に日本語標準のものとしてjsarticle等があります。
    \item プリアンブル:スタイルファイル(後述)の読み込み、自作命令(マクロという, 後述)の定義等を行う場所です。
    \item 本文:begin document, end documentの2つで囲まれた領域に、書類の本文を記述していきます。
\end{enumcircle}

\point*[running head]{テキストを打つ}
Googleドキュメント等
\point*[running head]{命令}
すたいるふぁいる
\point*[running head]{余白・改行}
\begin{framebox-simple}{改行}
    \begin{reitemize}
        \item \verb|\\|:強制改行
        \item 一行空白行を挿入すると改行されます
    \end{reitemize}
\end{framebox-simple}

\section{PDFにする}
\point*[running head]{エラーが起きた時は}% 2.6

参考書
\end{multicols*}