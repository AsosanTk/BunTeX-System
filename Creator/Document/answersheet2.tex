\documentclass[paper=b4j, landscape, fleqn, jafontsize=9pt, jafontscale=1, head_space=25.5mm, foot_space=15mm, fore-edge=14.5mm, gutter=14.5mm, hanging_punctuation, baselineskip=7mm]{jlreq}
\usepackage{../../Resources/BunTeX/buntexc}
\usepackage{../../Resources/BunTeX/buntexd}
\renewcommand{\lefttitle}{文付装飾委員会 アンケート}
\renewcommand{\righttitle}{第~1~回委員会時配布}

\newcommand{\visible}{visible}
\newcounter{mondai}
%
\newtcbox{\kasen}[1][]{boxrule=-1pt,bottomrule=.5pt,sharp corners, colback=white,\visible,nobeforeafter,tcbox width=forced left,box align=base,\visible,#1} %環境を定義。プリアンプルでも本文でも良い。

\newenvironment{question}[1]{
\addtocounter{mondai}{1}
{\selectsize{15pt}{15pt}\bfseries\noindent\arabic{mondai}}\hspace{1\zw}{\selectsize{10pt}{10pt}#1}\\
}{\vspace{3mm}}


\begin{document}
\begin{answersheet}
\begin{center}
    {\selectsize{15pt}{15pt}\textgt{
        \Helvetica{
            %高~2~英語 実践演習\\
            %前期第~2~週
            \noindent 第~1~回装飾委員会会合\\\noindent 初回アンケート
        }
    }}
    \vskip3mm
\end{center}

\begin{question}{自己紹介}
\noindent (1)~~ はじめに、学年・クラス・名前(ふりがな)を記入してください。\\
\noindent (2)~~ 所属している部活動・同好会・装飾委員会以外の委員会、もしくは装飾委員以外に文化祭に関して決まっている役職/仕事を教えてください。
\end{question}

\noindent (1)~~ \kasen{\hspace{2.5\zw}年\hspace{4\zw}組\hspace{23\zw}(\hfill)}\\
\noindent (2)~~ \kasen{}\\

\begin{question}{装飾委員会に関して}
\noindent (1)~~ 以前に装飾委員を務めた経験があるかを選択してください。ある場合は装飾委員の経験年数を教えてください。\\
\noindent (2)~~ 今回装飾委員になった経緯を1〜3の中から選択してください。\\
\noindent (3)~~ 正直やる気は...
\end{question}

\noindent (1)~~ \kasen{経験有無: ある ・ ない ~/~ 経験年数:   年}\\
\noindent (2)~~\hspace{1.5\zw}\egg{1} 第一希望だったので立候補  \egg{2} 第一希望ではないが立候補  \egg{3} じゃんけんで負けた\\
\noindent (3)~~\hspace{1.5\zw}\egg{1} とてもある  \egg{2} ある  \egg{3} 普通  \egg{4} あまりない  \egg{5} まったくない\\
    
\begin{question}{作業等に関して}
\noindent (1)~~ 1週間のうちで作業に参加できる曜日があれば教えてください(現時点でわかっている範囲で大丈夫です)。\\
\noindent (2)~~ 装飾でやりたいこと・目指したいことがあれば教えてください。\\
\noindent (3)~~ 校門装飾の概形の現時点での希望を教えてください。\\
\noindent (4)~~ メーリスに使用するメールアドレスを書いてください。\\
\noindent (5)~~ その他、質問や要望等があればお願いします。
\end{question}

\noindent (1)~~\hspace{1.5\zw}\egg{1} 月  \egg{2} 火  \egg{3} 水  \egg{4} 木  \egg{5} 金  \egg{6} 土  \egg{7} 日\\
\noindent (2)~~\hspace{1.5\zw}\egg{1} 達成感を得たい  \egg{2} 木工スキルを身に付けたい  \egg{3} 優しい先輩に褒められたい\\
~~~~~~~\egg{4} 差し入れが欲しい  \egg{5} 木工作業がしたい  \egg{6} 充実感を得たい\\
~~~~~~~\egg{7} マステが上手くなりたい  \egg{8} 作りたいものがある  \egg{9} ものづくりがしたい\\
~~~~~~~\kasen{その他:}\\
\noindent (3)~~\hspace{1.5\zw}\egg{1} ゲート  \egg{2} ツインタワー\\
\noindent (4)~~\kasen{}\\
\noindent (5)~~ \kasen{}\\~~~~~~~\kasen{}\\~~~~~~~\kasen{}\\
\begin{flushright}
    ご協力ありがとうございました。
\end{flushright}
\responsibility{麻生}{装飾部門長}
\end{answersheet}
\end{document}