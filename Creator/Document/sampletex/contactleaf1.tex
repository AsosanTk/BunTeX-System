\pagestyle{empty}
\contactleaf*[running head]{第1回デコ責会議 連絡リーフ}

\begin{multicols*}{2}
\setlength{\columnseprule}{0pt}
\contactleafsection*[running head]{添付資料一覧}
\begin{enumsquarebrackets}
    \item 総務部門連絡
    \item 提出物作成上の注意事項
    \item 高2文実メールアドレス・部門一覧表
    \item デコ責カード
    \item 参加申込書兼第1次中間報告書
\end{enumsquarebrackets}
※ 全ての書類が折り込まれているかご確認ください。\\
\noindent ※ 各書類のタイトルボックスの上には、配布元の部門と、<3桁>~\boldtext{-}~<4桁>で構成される書類番号が記されています。

\contactleafsection*[running head]{別紙配布資料一覧}
\begin{enumsquarebrackets}
    \item デコ責カード(解答用紙)
    \item 参加申込書兼第1次中間報告書(解答用紙)
\end{enumsquarebrackets}
※ 別紙2枚が手元にあることを確認してください。

\newcolumn
\contactleafsection*[running head]{提出締切書類一覧}
\begin{enumsquarebrackets}
    \item デコ責カード(4/28(木) 締切)
    \item 参加申込書兼第1次中間報告書(5/10(火) 締切)
\end{enumsquarebrackets}
※ 「締切」は指定日付の\boldtext{最終下校時刻}までを指します。23:59に出されたところでどうすることもできません。ご協力ください。

\contactleafsection*[running head]{次回連絡}
\begin{reitemize}
    \item 次回のデコ責会議は、\boldtext{\uwave{5/10(火) 12:40〜~@OS}}で行います。
    \item デコ責会議は、本日を除いて基本的には\boldtext{火曜日}に行います。特に2学期は毎週行いますので、忘れずに参加するようにしてください。
\end{reitemize}

\end{multicols*}