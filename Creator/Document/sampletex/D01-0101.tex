\newpage
\pagestyle{booklet}
\inserttitle*[running head]{総務~/~D01-0101}[総務部門連絡]
\begin{multicols*}{2}
 \\
\vspace{-15mm}
\section{文化祭の日程}
\indent 今年度の文化祭は10月28日(金)〜10月30日(日)に開催されます。約半年間よろしくお願いします。

\section{デコ責会議}
\indent 今年度のデコ責会議は原則、火曜日12:40よりOSにて開催します。\\
\indent 1学期は5月10日に第2回、5月31日に第3回、7月5日に第4回、7月20日に第5回となります。\\
\indent 2学期は原則として毎週火曜日に開催する予定ですが、祝日等の都合により変更になる場合があります。
加えて9月1日に第6回を、11月18日に第15回を例外的に開催する予定です。

\section{デコ責会議配布書類}
\indent 今年度はデコ責会議で配布する書類を冊子にまとめて配布します。
提出物の記入用紙は対応するページに折り込まれています。\\
\indent 乱丁落丁などありましたら速やかに申し出るようにしてください。
デコ責会議終了後の該当回の冊子の再発行は原則として受け付けませんので、大切に保管してください。
なお、提出物の記入用紙は紛失した場合、上本部にて再発行いたします。

\section{Classroom参加のお願い}
\indent 今年度もデコ責用のClassroomを作成しました。
デコ責会議で配布した書類のpdfデータを掲載したり、緊急的な連絡に使用しますので各団体のデコ責は2名とも必ず参加するようにしてください。
\checking{Class code:\boldtext{dpuu2ai}}

\section{デコ責講習会}
\indent 中1中2の全デコ責及び参加を希望するデコ責を対象に、4月26日15:20〜16:20に生物講義室にて、デコ責講習会を開催します。\\
\indent 4月から文化祭までの作業の流れ、デコ責を初めて務める際によく起こるミスなどを説明します。
その上で、1学期に行うべきことについて詳しく説明していきます。\\
\indent 忙しい中とは思いますが、デコ責は非常に作業が多岐にわたるため、未経験者にとっては非常に苦労の多い役職です。
今後の作業をスムーズに進めるためにも、対象となっているデコ責は必ず参加するようにしてください。どうしても都合がつかない場合は連絡をください。

\section{問い合わせ}
\indent 文化祭準備に関する相談がありました、上本部まで来ていただければ対応します。1学期は基本的に17:00前後までは対応可能です。\\
\indent 何らかの事情で上本部まで相談に来ることが難しい場合、以下のメールアドレスに要件を送っていただければ対応いたします。
\checking{\boldtext{bunkasai2022@tsukukoma-gafe.org}}

\responsibility{松丸}{総務}
\end{multicols*}