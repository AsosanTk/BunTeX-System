\newpage
\pagestyle{booklet}
\inserttitle*[running head]{総務~/~D01-0105}[参加申込書兼第1次中間報告書]

\begin{multicols*}{2}
 \\
\vspace{-15mm}
\section{デコの内容}
\indent デコの内容に関する質問に、期限内に最終的な結論を出すことは困難ですし、十分な議論をする時間がないことは承知しています。
重要なことは結論を急がず、デコの構成員の意見を拾いながら議論を深めていくことです。\\
\indent ですから、この第1次中間報告書に対して完璧な回答をする必要はまったくもってありませんし、団体内での議論の習熟度に応じて適切に回答していただければ十分です。

\point*[running head]{デコの目標}
\indent デコのジャンルやテーマについて話し合う前に、まずはデコの目標についてよく議論してみてください。
デコの方向性とは、例えばデコをどんな人に見てほしいかであったり、あるいはデコを通じてどんなことを伝えたいかであったりします。
端的に言えば、なぜデコを作るのかということをよく見つめてほしいということでこの質問を設定しました。\\
\indent なお、この質問に関する回答は「AとBの2案で迷っていて、何月何日に投票をとろうと思っている」など最終的な結論に達していない回答でも構いません。

\point*[running head]{デコの興味範囲}
\indent なぜデコを作るのかの議論と同時並行的に、団体の構成員はお互いにどういうことに興味があるのか、どういうことをやってみたいと思っているのかも話し合ってみてください。
デコの目標とデコの興味範囲が把握できれば、おのずとデコの方向性は定まっていくはずです。\\
\indent なお、この質問に関する回答は「AとBの2案で迷っていて、何月何日に投票をとろうと思っている」など最終的な結論に達していない回答でも構いません。


\section{誓約文}
\indent 誓約文をよく読み同意の上で、署名してください。解答は別途配布の用紙(提出物番号:D01-0105A)に記入してください。

\responsibility{松丸}{総務}
\end{multicols*}