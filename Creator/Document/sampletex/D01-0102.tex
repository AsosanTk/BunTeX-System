\newpage
\pagestyle{booklet}
\inserttitle*[running head]{総務~/~D01-0102}[提出物作成上の注意事項]
\begin{multicols*}{2}
 \\
\vspace{-15mm}
\section{形式面に関して}
\point*[running head]{記入に関して}
\indent 手書きで記入する際には、必ずボールペン(フリクション不可)を用いてください。\\
\indent 配布した記入用紙に手書きで記入せず、PC等で回答をまとめたものを作成し印刷したものを記入用紙に添付して提出することも受け付けます。
ただし、\boldwave{全ての質問に回答し}、\boldwave{必ず記入用紙のデコ責署名欄・顧問署名欄は埋めるようにしてください}。
\point*[running head]{期限}
\indent 各提出物は文実が各デコの様子を把握し、適切なサポートを提供するために必要不可欠です。必ず期限を守るようにお願いします。
\point*[running head]{提出場所}
\indent 基本的に各提出物の提出先は上本部前のピジョンボックスとなります。
ただし一部の提出物については手渡しでの提出が必須となりますのでご注意ください。
\point*[running head]{団体名}
\indent 団体名はデコタイトル(デコ名)とは異なります。\\
\indent 団体名はパンフレットに掲載するものと同じ正式名称を記入してください。\\
HR団体の正式名称は「1-A HR」であり、「1A」や「1-A」でないことに注意してください。
\point*[running head]{顧問署名欄}
\indent しっかりと内容を確認してもらったうえで、顧問署名をもらうようにしてください。
顧問の研修日で署名がもらえなかったため、期限を守れないなどということがないようによく気をつけてください。

\section{内容面に関して}
\point*[running head]{適切な回答}
\indent 提出物は文実が各デコに適切なサポートを行う基盤となるものです。
「詳細に」とあれば詳細に、字数制限があれば守ってというように質問に合わせて、適切にお答えください。
質問の意図などについて疑問がある場合は、上本部まで質問に来ていただければ対応いたします。お気軽にお越し下さい。\\
\indent 当たり前ですが、質問に対しデコの実態と反する回答を行った場合、文実が各デコに適切なサポートを行えず、
安全管理などが難しいことから最悪の場合、デコ停止処分などに踏み切る可能性があります。
提出物には必ずデコの実態に即した回答を記入し、訂正がある場合には直ちに申請書を提出するようにしてください。
\point*[running head]{「未定」~}
\indent 未定と記入することはないようにしてください。未定と回答することは、実質的に期限を守っていないのと変わりません。
原則として、期限までにデコ内で質問に関する回答を話し合うなどして、まとめるようにしてください。
ただし、中間報告書のデコ内容に関する記入のみ、「AとBの2案で迷っていて、何月何日に投票をとろうと思っている」など
最終的な結論に達していない回答で構いません。

\section{申請書について}
\point*[running head]{申請書とは}
\indent 申請書とは、デコを作る上で特殊なことをする場合、または提出した書類から計画を変更する場合に必要となります。
\point*[running head]{申請書の提出方法}
\indent 上本部前のピジョンボックスに白紙の申請書が入っていますので、必要事項を記入の上、担当総務まで提出してください。
担当総務が決定するまでは、松丸に提出するようにしてください。

\responsibility{松丸}{総務}
\end{multicols*}