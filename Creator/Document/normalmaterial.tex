\documentclass[paper=b4j, landscape, fleqn, jafontsize=8pt, jafontscale=1, head_space=25.5mm, foot_space=15mm, fore-edge=14.5mm, gutter=14.5mm, hanging_punctuation]{jlreq}

\usepackage{../../Resources/BunTeX/buntexc}
\usepackage{../../Resources/BunTeX/buntexd}

\renewcommand{\lefttitle}{文付審査委員会 会合資料}
\renewcommand{\righttitle}{第~1~回}

\begin{document}
\begin{normalmaterial}
\materialtitlea*[running head]{第1回審査委員会}
\materialpart*[running head]{審査委員会へようこそ}
\section{審査委員会}
\point*[running head]{審査委員会}
審査委員会は、文化祭実行委員会(文実)に付属する委員会のうちの1つで、審査委員長、副委員長および高1代表が執行部として運営を行います。顧問は生物科の内山先生です。

\point*[running head]{執行部}
\begin{reitemize}
    \item 審査委員長:尾崎瑛彦 (2-2) 
    \item 審査副委員長:巻島快世 (2-2)
    \item 審査副委員長:鈴木康平 (1-2)
    \item 高1代表:米澤優斗 (1-2)
\end{reitemize}
\point*[running head]{主な仕事}
審査委員会の主な仕事は、文化祭当日にいくつかの部門にわかれて各団体を公正に審査して回り、最優秀賞や優秀賞などの賞をつけることです。また、ポスターなどの宣伝物の審査も行います。そして、来場者が投票をする大衆賞の管理も同時に行います。

\section{審査}
\point*[running head]{審査の目的}
\checking{モチベーションの向上}\\
審査とは、文化祭において各デコとその宣伝物を公正に審査・評価し、優秀なものに賞を与えるシステムのことで、文化祭へのモチベーションを高め、デコ、ひいては文化祭全体の質を高めることを目的として行われています。\\
このように、「生徒が生徒のつくったものを審査し、質を高める」という仕組みがある学校は珍しく、我々はいわば「\boldwave{文化祭における自治}」を担っているといえます。\\
\checking{文化祭へのフィードバック}\\
審査委員会は、文化祭後にもフィードバックをする活動があることが特徴となっており、これは、「\boldwave{来年度以降の文化祭に対しても大きく貢献する}」ことも審査の大切な目的であることを表しています。

\point*[running head]{注意点}
審査委員は比較的忙しい仕事です。そのため、\boldwave{当日忙しくなるデコ責、キャスト(公演への出演)、高校監査委員との兼任は絶対にできません。}また、当日は必ず審査の仕事を優先し、担当時間外もなるべく他のシフトを入れないでください。

\section{委員会についてのお願い}
\point*[running head]{会合の予定について}
基本的に木曜日の12:40から生物講義室で会合を行う予定です。ただし、今後の動向により突発的に会合を開く可能性があります。

\point*[running head]{シフトについて}
きちんと出席してくれる人が損しない環境を作るため、\boldwave{必ず参加する必要のある会合にどの程度出席したかに応じて文化祭当日の仕事の量が変動する場合があります}。なお、やむを得ず参加できない場合は、「sinsatkfes@gmail.com」まで名前を添えて連絡するか、学年代表の人を通じて連絡するようにしてください。(事前・事後どちらでも可)連絡があった場合、欠席を不利に扱うことはありません。

\point*[running head]{出欠確認}
委員会の時間を有効的に使うため、できるだけ早いうちに出席確認を終わらせたいと考えています。そのため、\boldwave{必ず委員会開始時に資料を取った状態で席についているようにしてください}。資料は教卓の上に置いておく予定です。

\section{情報共有について}
\point*[running head]{資料の共有}
今後、委員会を欠席した方や資料を紛失した方向けに、委員会で配布した資料を審査委員会のClassroomで共有していく予定です。「紙の資料が欲しい」という方や、「資料について詳しい話を聞きたい」という方は、執行部に教えてください。\\
\boldwave{審査委員会のClassroomのコードは「qtlx5e4」です}。

\point*[running head]{連絡方法}
今後、集会で委員会に関する連絡を行うだけでなく、\boldwave{連絡黒板とClassroomに委員会日程を掲示しますので、必ずチェックするようにしてください}。

\section{役職について}
\point*[running head]{学年代表}
その学年の委員全員に関する連絡(委員会日程の確認や欠席連絡)を受け持つ係です。\\
学年代表決定後、\boldwave{口頭以外の連絡手段について今この場所で決めてください。}各学年でスマートフォンの使用に関するルールがあると思うので、実際に用意するのは今すぐでなくても大丈夫です。(後日の委員会で用意が完了したか確認します。)\\
なお、連絡手段に関しては、同じく後日の委員会で学年代表の方にお渡しする@tsukukoma-gafe.orgのアドレス一覧表を利用しても良いですし、LINE等のメッセージサービスを利用しても大丈夫です。

\point*[running head]{中学代表(学年代表兼任)…中3から1名}
中3の委員全員に関する連絡だけでなく、中1・中2の学年代表を支えます。高校生に話しかけづらい中学生の委員は、中学代表を経由して執行部に伝えることが可能です。

\point*[running head]{副委員長…高1から1名、高2から1名}
委員会執行部として、委員長の補佐や審査制度の立案をします。

\point*[running head]{委員長(学年代表兼任)…高2から1名}
今後の会合で議長を務めます。また、委員会への連絡(放送・連絡黒板・集会)や審査制度の立案をします。

\section{最後に}
先述したように審査にはそれなりの責任が伴いますが、その分得られるものも大きいと考えています。(将来いろんな道で活躍するであろう)筑駒生の作ったものを見て、それについて考え、議論を交わせるというのはとても貴重な経験だと思いますし、「審査する」ことは筑駒文化祭について深く知るきっかけにもなると思っています。\\
審査に興味を持ってもらい、数年後に執行部となる人が一人でも出てくれると非常に嬉しいです。皆さんが「審査委員になってよかった」と思えるような委員会活動を実現すべく、執行部一同尽力してまいります。

\section{事務連絡}
\begin{reitemize}
    \item アンケートを提出した人から解散です。
    \item 第2回審査委員会は、\boldwave{4/28(木)の12:40から生物講義室}で行います。
    \item Google Formでも実施するアンケートの紙版を配布します。回答内容によって当日のシフトが変動することはありませんので、率直な意見を回答していただけるとありがたいです。
\end{reitemize}
\responsibility{尾崎}{審査}
\distribution{2022/4/19(火)}
\end{normalmaterial}
\end{document}