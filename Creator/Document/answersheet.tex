\documentclass[paper=b4j, landscape, fleqn, jafontsize=9pt, jafontscale=1, head_space=25.5mm, foot_space=15mm, fore-edge=14.5mm, gutter=14.5mm, hanging_punctuation, baselineskip=7mm]{jlreq}
\usepackage{../../Resources/BunTeX/buntexc}
\usepackage{../../Resources/BunTeX/buntexd}
\renewcommand{\lefttitle}{文付審査委員会 アンケート}
\renewcommand{\righttitle}{第~1~回委員会時配布}




\begin{document}
\begin{answersheet}
\begin{center}
    {\selectsize{15pt}{15pt}\textgt{
        \Helvetica{
            %高~2~英語 実践演習\\
            %前期第~2~週
            \noindent 第~1~回審査委員会\\\noindent 初回アンケート
        }
    }}
    \vskip3mm
\end{center}

\begin{question}{自己紹介}
\noindent (1)~~ はじめに、学年・クラス・名前(ふりがな)を記入してください。\\
\noindent (2)~~ 所属している部活動・同好会・審査委員会以外の委員会を教えてください。
\end{question}

\noindent (1)~~ \kasen{\hspace{2.5\zw}年\hspace{4\zw}組\hspace{23\zw}(\hfill)}\\
\noindent (2)~~ \kasen{}\\

\begin{question}{審査委員会に関して}
\noindent (1)~~ 以前に審査委員を務めた経験があるかを選択してください。ある場合は審査委員の経験年数と、その時に所属していた部門を教えてください。\\
\noindent (2)~~ 今回審査委員になった経緯を1〜3の中から選択してください。\\
\noindent (3)~~ 正直やる気は...
\end{question}

\noindent (1)~~ \kasen{経験有無: ある ・ ない ~/~ 経験年数:   年  所属部門:}\\
\noindent (2)~~\hspace{1.5\zw}\egg{1}  第一希望だったので立候補  \egg{2} 第一希望ではないが立候補  \egg{3} じゃんけんで負けた\\
\noindent (3)~~\hspace{1.5\zw}\egg{1} とてもある  \egg{2} ある  \egg{3} 普通  \egg{4} あまりない  \egg{5} まったくない\\
    
\begin{question}{今年の文化祭に関して}
\noindent (1)~~ 現時点で審査委員以外に文化祭に関して決まっている役職/仕事があれば教えてください。\\
\noindent (2)~~ その他、何かご質問等があれば自由に記述して下さい。
\end{question}

\noindent (1)~~ \kasen{}\\
\noindent (2)~~ \kasen{}\\~~~~~~~\kasen{}\\~~~~~~~\kasen{}\\
\begin{flushright}
    ご協力ありがとうございました。
\end{flushright}
\responsibility{尾崎}{審査}
\end{answersheet}
\end{document}