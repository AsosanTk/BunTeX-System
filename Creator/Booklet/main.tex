\documentclass[book, twoside, paper=b5j, fleqn, jafontsize=9pt, jafontscale=1, head_space=22mm, foot_space=9mm, fore-edge=16mm, gutter=25mm, hanging_punctuation]{jlreq}

\usepackage{../../Resources/BunTeX/buntexc}
\usepackage{../../Resources/BunTeX/buntexb}

%Tikzで遊んでみる
%% 最後に寸法調整を忘れないこと!! %%

\begin{document}
% 表紙1
\pagestyle{empty}
\covera{高1数学 後期}{数学発展講座~\textsf{I}~/~\textsf{II} 第2部}{鉄緑会}{数学科}{21B4-1003}

\newpage
\vbox to 15truemm{\vskip0truemm
    \hbox to 144truemm{\hskip0truemm
        \hfill\gothic{\selectsize{15truept}{15truept}クラス~}\underline{\hspace{23truemm}}\hspace{5truemm}\gothic{\selectsize{15truept}{15truept}氏名~}\underline{\hspace{70truemm}}\\
        \hfill\gothic{21G4-0101}
    \hss}\vss}
\newpage
\chapter{原子とイオン}
\section{おはよう}
\subsection{こんにちは}
\keypoint{キーポイント}
\subkeypoint{サブキーポイント}
\begin{itemize}
    \item こんにちは
    \item おはよう
    \item また来年
    \item おはよう
\end{itemize}
\begin{enumerate-brackets}
    \item こんにちは
    \item おはよう
    \item また来年
    \item おはよう
\end{enumerate-brackets}
\begin{enumerate-circle}
    \item こんにちは
    \item おはよう
    \item また来年
    \item おはよう
\end{enumerate-circle}
\begin{enumerate-square}
    \item こんにちは
    \item おはよう
    \item また来年
    \item おはよう
\end{enumerate-square}


\uline{テキストのアンダーライン}、\boldtext{\uline{太字の下線}}\\
\mline{取り消し線、変なこと言うなよ}\\
\uwave{なみなみ線たのしいね☆}
\udotline{点線の下線みたい}
なんとなくの\kenten{わくわく}\\
\ltjruby{九|去|法}{きゅう|きょ|ほう}\\
\textgt{\ltjruby{九}{きゅう}\ltjruby{去}{きょ}\ltjruby{法}{ほう}}こんなかんじかな

\begin{framebox-note}{サンプル}
    これが補足だよ☆
\end{framebox-note}
\begin{framebox-ref}{参考のタイトル}
これが参考だよ☆
\end{framebox-ref}



\newpage
\pagestyle{normal}
\vspace*{9truemm}
\gothic{\hspace*{-1.5\zw}【高1化学 週別カリキュラム表】\hfill 《2021年度 月曜日》}
\vspace{2truemm}
\begin{center}
    \begin{tabularx}{140truemm}{Ic|c|c|C|cI} \noalign{\global\arrayrulewidth1pt} \hline \noalign{\global\arrayrulewidth.3pt}
        \gothic{日付} & \gothic{曜日} & \gothic{週} & \gothic{授業内容} & \gothic{総復習テスト} \\ \noalign{\global\arrayrulewidth1pt} \hline \noalign{\global\arrayrulewidth.3pt}
        \gothic{1月10日} & \gothic{(月)} & \gothic{第01週} & \gothic{原子とイオン} & \gothic{-} \\ \noalign{\global\arrayrulewidth.3pt} \hline 
        \gothic{1月10日} & \gothic{(月)} & \gothic{第01週} & \gothic{原子とイオン} & \gothic{-} \\ \hline
        \gothic{1月10日} & \gothic{(月)} & \gothic{第01週} & \gothic{原子とイオン} & \gothic{-} \\ \hline
        \rowcolor[gray]{0.8} \gothic{1月10日} & \gothic{(月)} & \multicolumn{3}{cI}{\gothic{第4期休講期間}} \\ \noalign{\global\arrayrulewidth1pt} \hline
    \end{tabularx}
\end{center}
\vspace{2truemm}
\gothic{※W1・W2とはダブル授業の1回目・2回目を指します。}

\newpage
\section{こんにちは}
\nosubsection
\boldtext{自然界⑵}には、さまざまな物質が存在するが、その成分を分解してゆくと、どのような成分にたどり着くのか。
この問題は、古代からさまざまな哲学者・科学者によって議論されてきた。
\section{展示を行う場合}
\subsection{調べ物とは}
展示には色々な形態があります。調べたことを模造紙に書いてパネルに貼っていく(場合によってはパネルに直接書く)、模型を作って展示する、来場者に体験をしてもらう、といったものが挙げられるでしょう。これらの複数を組み合わせても良いでしょう。しかし、いずれの方法をとるにしても、何らかの形でものを調べる・調査するということが必要になります。
調査をする方法にも色々あり、図書館や本屋で文献を探す、新聞や雑誌の記事を探す、インターネットから情報を集めるといった方法のほか、専門家にアポを取って直接お話をいただくという手もあります。\\これらの中でどの調査方法をとるかによって、展示のクオリティーや、そのデコが得るものが大きく違ってきます。
本章では、調査の中でも特に重要と思われる文献調査のコツを紹介していきます。ここに書かれたことを参考にして、各デコで実りある調査・研究をし、その成果を文化祭で多くの来場者に発表していただければと思います。ただし、本章の記述には主観も多分に含まれていますので、鵜呑みにはせず、各デコの判断で活用してください。なお、本章は、小冊子「調査のための文献探索案内」(本校国語科編、2006.09)を基に書き起こされたものです。
\subsection{文献調査の意義}
某高校の文化祭でとある展示に足を運んだ時、展示されていた文章にふと違和感を覚え、携帯電話でWikipediaの当該ページを覗いてみました。するとどうでしょう、展示されていた文章はWikipediaのまる写しだったのです。
これは流石に極端ですが、インターネットの内容を貼り合わせてきただけのものを展示するだけでは、来場者の関心を惹きつけるどころかその目をごまかすことすらままなりませんし、デコが得るものも少ないことでしょう。
また、よく言われることですが、ウェブサイトの内容は必ずしも正しくありません。インターネットの場合は誰が書いているかも分からず、従って書き手は自分の書いたものに対して無責任であると言って良いでしょう(例えば、一般の人が専門家を騙り、デタラメなことを書くことも可能なのです)。
一方、文献の場合は著者が明示されており、書き手はその文献の内容に対する責任を認識しているはずです。確かに文献にも誤りはあるかもしれませんが、その数はインターネットよりも大幅に少ないは
\subsection{調べ物とは}
展示にはいろいろな形態がありますね〜!

\section{原子}
\subsection{物質の構成成分をめぐる歴史}
全ての原子の中で最も簡単なのは水素原子であり、1個の用紙からなる原子核と1個の原子
から構成されている。また、ヘリウム原子は、2個


さてさてこんなことしてる場合じゃないんですよ先生。明日はそうふくですよ?まだ勉強してないんだけど。困ったね全く。




{\HiraKakusix\fontsize{15truept}{15truept}\selectfont\noindent 高1化学 授業概要({\HiraKakusixE 2021}年度)}

\begin{multicols*}{2}
    \aboutsection{学習目標}
    \indent 化学基礎講座においては、高1の4月から高2の3月にかけての15ヶ月間で、高校化学を一通り一周します。\\
    \vspace{5truemm}
    \aboutsubsection{⑴化学基礎講座}
\end{multicols*}


上演の許可が下りた場合、次は改変の許可を取ってください。\\
文化祭で上演する場合、長い脚本を短くしたり、台詞を言いやすいように変えたり、ということが行われると思いますが、それが著作者人格権の侵害になる可能性があります。
演じやすいように自由に変えてかまわない、といわれた場合には問題はありませんが、改変をするなと言われることもあります。その場合、上演時間が長くなる等の問題が起こりますので、こちらもデコの中で意見統一をしておきましょう。
上演、改変の許可が取れたら書面で正式に申請してください。テンプレートは最後に載せてあります。
\\
(5)改変終了後に許可を取る\\
改変の許可において最も多いのは「見せてもらわないと分からない」という返答でしょう。\\
その場合、脚本を上演用に書き直し、著作者に見せる必要があります。本番までに時間がありますから暫定的なものになるでしょうが、それを郵送する形になると思います。郵送を7月の終わりまでに行えればあとの流れがスムーズです。
脚本の書き換えが終わったら、著作者に電話を入れてこれから郵送する旨を伝えてください。電話の際、以降本番までに細かい変更が多くありそうだ、ということも伝えておきましょう。
また「文化祭に向けて、非常に短い期間で制作を行う」ということを説明するとよいでしょう。プロの世界は余裕のあるスケジュールで動いていることも多く、誤解を防ぐためにも事情をよく知っておいてもらう必要があります。
具体的な案が先方から出ない場合、8・9月末に\footnotemark[1] 変更箇所の一覧を送るという提案をして話し合ってください。そして看過できない変更のあった場合には連絡をお願いします、といってください。
言うまでも無いことですが、文化祭本番の日付も伝え忘れないように。
\footnotetext[1]{tomはギリシャ語で「切る」を意味し、それを接頭辞a-で否定することで、atomで「着ることができない」という意味になる。tomが「着る」を意味している英単語としては、anatomy(解剖学)などがある。}
作者も忙しいのであまり頻繁に連絡をして邪魔をしないようにしましょう。\\
10月中旬には上演用の脚本が確定すると思います。その際は確定版であることを伝えて著作者に送り、本番に臨みましょう。
\\
(6)お礼をする\\
文化祭が終了したら著作者・著作権者にお礼の挨拶をしましょう。お礼の仕方には手紙を送るほかビデオに撮って郵送するなど様々な方法があると思いますが、最善の方法は各デコで判断してください。何より大切なのは相手方との信頼関係を築くことで

\newpage
\pagestyle{empty}
\vfill
\begin{center}
    \setlength{\tabcolsep}{5truemm}
    \renewcommand{\extrarowheight}{2truemm}
    \begin{tabular}[H]{c} \toprule
        高1化学\\
        {\fontsize{14truept}{14truept}\selectfont\kintou{44truemm}{化学基礎講座}}\\
        \rule[0truemm]{0truemm}{8truemm}\\
        \kintou{12truemm}{編者}\hspace{8truemm} \kintou{24truemm}{鉄緑会化学科}\\
        \kintou{12truemm}{発行者}\hspace{8truemm} \kintou{24truemm}{鉄緑会}\\
        2021~年~1~月~1~日    初版第1刷発行\\\bottomrule
        \multicolumn{1}{l}{非売品}\\
        \multicolumn{1}{l}{許可なしに転載、複製、転売、譲渡することを禁じます。}\\
        \rule[0truemm]{84.5truemm}{0truemm}\\
    \end{tabular}
\end{center}
\vspace{10truemm}%??
\newpage
\end{document}
